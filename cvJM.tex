\documentclass[a4paper,12pt]{article}
\usepackage{marvosym}
\usepackage{fontspec}
\usepackage{xunicode,xltxtra,url,parskip}
\RequirePackage{color,graphicx}
\usepackage[usenames,dvipsnames]{xcolor}
\usepackage[big]{layaureo} 				
\usepackage{supertabular}
\usepackage{titlesec}	
\usepackage{verbatim}
\usepackage{nth}
\usepackage{multicol}
\usepackage{longtable}
\usepackage{hyperref}
\usepackage{ragged2e}
\definecolor{linkcolour}{rgb}{0,0.2,0.6}
\hypersetup{colorlinks,breaklinks,urlcolor=linkcolour, linkcolor=linkcolour}

\geometry{lmargin=1.5cm, rmargin=1.5cm}
% \defaultfontfeatures{Mapping=tex-text}
% \setmainfont[
% SmallCapsFont = Fontin-SmallCaps.otf,
% BoldFont = Fontin-Bold.otf,
% ItalicFont = Fontin-Italic.otf
% ]
% {Fontin.otf}
\titleformat{\section}{\Large\scshape\raggedright}{}{0em}{}[\titlerule]
\titlespacing{\section}{0pt}{3pt}{3pt}
\usepackage[absolute]{textpos}

\begin{document}
\font\fb=''[cmr10]''

%--------------------TITLE-------------
\par{\centering
		{\Huge Andrea \textsc{De Polis}
	}\par}

%--------------------SECTIONS----------------------------------
\section{Personal Data}
\renewcommand{\arraystretch}{0.75}
\begin{tabular}{ll}
        \textsc{email:} %& \href{mailto:andrea.de-polis@strath.ac.uk}{andrea.de-polis@strath.ac.uk} \\
            & \href{mailto:andrea.depolis@gmail.com}{andrea.depolis@gmail.com}\\
    \textsc{Website:}   & \url{https://andreadepolis.github.io/}            
                        
\end{tabular}
\section{Research Interests}
Primary fields: Macroeconometrics, Financial Econometrics, Macroeconomics\\[.5em]
Secondary fields: Bayesian Econometrics, Forecasting, Asset Pricing 

%\section{Current Position}
%\begin{tabular}{ll}
%2022- & \textsc{Economist}\\
%& Research Department, \textbf{Fulcrum Asset Management}
%\end{tabular}
\section{Current Position}

\begin{tabular*}{\linewidth}{@{}l@{\extracolsep{\fill}}l}
\textbf{Research Economist} & Oct 2025 - \phantom{Sept 2024}\\[.2em]
Monetary Policy Division\\
Banco de Espa\~na%\\[1em]
\end{tabular*}

\section{Professional Experience}
\begin{tabular*}{\linewidth}{@{}l@{\extracolsep{\fill}}r}
\textbf{Postdoctoral Research Associate} & Jan 2024 - Sept 2025\\[.2em]
Economic Statistics Centre of Excellence (ESCoE)\\
Univeristy of Strathclyde\\[1em]
%\textbf{Research Director} (part-time since Jan 2024) & Apr 2024 - \phantom{Sept 2024}\\[.2em]
%\textbf{Research Economist} & Sept 2022 - Mar 2024\\[.2em]
\textbf{Research Visiting} & May 2025 \phantom{-Sept 2024}\\
Central Bank of Finland\\[1em]
\textbf{Research Economist} & Sept 2022 - Aug 2025\\[.2em]
Fulcrum Asset Management\\[1em]
\textbf{Research Visiting} & May 2023 \phantom{-Sept 2024}\\
Macro Research Group\\
Federal Reserve Bank of Chicago\\[1em]
\textbf{Ph.D. Research Assistant} & Jan 2022 - Aug 2022\\[.2em]
DG-Research\\
European Central Bank\\[1em]
\textbf{Senior Economist} & May 2021 - Dec 2021\\[.2em]
Now-Casting Economic, Ltd\\[1em]
\textbf{Research intern} & Sept 2018 - Jan 2019\\[.2em]
Monetary Policy and Economic Outlook Directorate\\
Bank of Italy
\end{tabular*}

\section{Education}
\begin{tabular*}{\linewidth}{@{}l@{\extracolsep{\fill}}l}
\textbf{Ph.D. in Finance and Econometrics} & 2017 - 2023 \\[.2em]
Supervisors: Prof. Ivan Petrella and Prof. Ana Galv\~ao\\
Viva Committee: Prof. Andrew Patton and Prof. Anthony Garratt\\
Warwick Business School, The University of Warwick \\
\\
\textbf{MSc in Economics}, Cum laude & 2015 - 2017\\[.2em]
Tor Vergata University of Rome\\
%&\href{https://www.dropbox.com/s/4td40tzwfrivuc9/Pres.zip?dl=0}{Thesis}: ``Evaluating Quantitative Easing in the Eurozone:\\
%& \hspace{3.5em} a Bayesian DSGE approach''\\
%& Supervisors: Prof. Luisa Corrado and Prof. Tommaso Proietti\\&\\
&\\
\textbf{BSc in Economics} & 2012 - 2014\\[.2em] 
Roma Tre University\\
%& Thesis: ``FDI: a step towards the SDGs''\\
%&\small Advisor: Prof. Salvatore Monni\\
\end{tabular*}
\newpage
\section{Research}
\subsubsection*{Publications}
\textit{\textbf{Modeling and Forecasting Macroeconomic Downside Risk}}\\ with Davide Delle Monache (Banca d'Italia) and Ivan Petrella (Collegio Carlo Alberto).\\ \textbf{Journal of Business \& Economic Statistics}, 42 (3), 1010 - 1025, 2024.\\[.2em]
\textit{Abstract}: We model permanent and transitory changes of the predictive density of U.S. GDP growth. A substantial increase in downside risk to U.S. economic growth emerges over the last 30 years, associated with the long-run growth slowdown started in the early 2000s. Conditional skewness moves procyclically, implying negatively skewed predictive densities ahead and during recessions, often anticipated by deteriorating financial conditions. Conversely, positively skewed distributions characterize expansions. The modeling framework ensures robustness to tail events, allows for both dense or sparse predictor designs, and delivers competitive out-of-sample (point, density and tail) forecasts, improving upon standard benchmarks.

\subsubsection*{Working Papers}
\textit{\textbf{The Taming of the Skew: Asymmetric Inflation Risk and Monetary Policy}},\\ with Leonardo Melosi (University of Warwick, EUI) and Ivan Petrella (Collegio Carlo Alberto).\\[.2em]
\textit{Abstract}: We document that inflation risk in the U.S. varies significantly over time and is often asymmetric. To analyze the first-order macroeconomic effects of these asymmetric risks within a tractable framework, we construct the beliefs representation of a general equilibrium model with skewed distribution of markup shocks. Optimal policy requires shifting agents’ modal forecast counter to the direction of inflation risks. We perform counterfactual analyses using a quantitative general equilibrium model to evaluate the implications of incorporating real-time estimates of the balance of inflation risks into monetary policy communications and decisions.\\[.5em]

\textit{\textbf{Time-Varying Skewness and Momentum Crashes}},\\ with Daniele Bianchi (Queen Mary University) and Ivan Petrella (Collegio Carlo Alberto).\\\textbf{Revision requested at the Review of Asset Pricing Studies}.\\[.2em]\textit{Abstract}:
The return on conventional momentum portfolios exhibits a predominantly negative, time-varying skewness, which deepens during momentum ``crashes''. This has important implications for the portfolio risk-return trade-off: the relationship between the expected return and volatility is time-varying and depends on conditional skewness. We explore the economic underpinnings of time-varying skewness by timing the capital exposure to a momentum portfolio in response to fluctuations in risk. The results show that a dynamic skewness-adjusted maximum Sharpe ratio strategy outperforms popular volatility targeting approaches. Finally, we show that momentum skewness cannot be fully reconciled with an asymmetric exposure to upside and downside market risk.\\[.5em]

\textit{\textbf{Testing for Conditional Skewness with Epsilon-Skew-t Distributions}}.\\[.2em]
\textit{Abstract:} I develop a parametric test to detect the presence of instability in the third moment of time series data. The test is based on the score function of the flexible \textit{epsilon-Skew-t} distribution, and belongs to the class of Lagrange Multiplier tests. The test presents appropriate asymptotic properties, as evaluated by means of an extensive Monte Carlo analysis. When applied to the three asset pricing anomalies of Fama and French (1993), the test points at an overwelimg evidence of conditional non-Gaussianity at the daily frequency, whereas weaker results are observed at the monthly frequency. These results should be taken as a warning of possible misspecification of asset pricing models based on symmetric likelihoods. \\[.5em]
\textit{\textbf{Exchange Rate Dynamics and Unconventional Monetary Policies: it’s all in the shadows}},\\ with Mario Pietrunti (Banca d'Italia).\\ %Bank of Italy Temi di Discussione (Working Paper) No.1231.\\[.5em]
\textit{Abstract:} In this paper we estimate an open economy New-Keynesian model to investigate the impact of unconventional monetary policies on the exchange rate, focusing on those adopted since the Global Financial Crisis in the euro area and in the United States. To this end we replace effective, short-term, interest rates with shadow rates, which provide a measure of the monetary stance when the former reach their effective lower bound. We find that since 2009 unconventional monetary policies significantly affected the dynamics of the euro-dollar exchange rate both in nominal and real terms: while the stimulus provided by the Fed prevailed between 2011 and 2014, contributing to the weakening of the dollar, in most recent years the depreciation of the euro mainly reflected the measures adopted by the ECB.

\subsubsection*{Work-in-progress}
\textit{\textbf{Common Risks and Common Gaps}}.\\[.2em]
\textit{Abstract:} The Okun's law puts forward a relation between the output and the unemployment gaps. I provide evidence of common non-linearities across the two quantities, highlighting common dynamics in higher order moments. I estimate a joint model for the dynamics of the marginal densities of the output gap and the unemployment gap, which can capture potential non-Gaussian features of the data through the time variation of the mean, variance and skewness. I postulate the Okun's relation to hold for the predictive densities by assuming common cyclical components for the moments. I document a considerable reduction in the uncertainty surrounding estimates of the natural rate of unemployment, or NAIRU, in the US, as compared to estimates based on symmetric models. Similarly, accounting for time-varying skewness of the output delivers estimates of the output gap that are less uncertain and more stable over time with respect to CBO projections.\\[.5em]

\textit{\textbf{The Ever-Changing Challenges to Price Stability}},\\with Leonardo Melosi (University of Warwick, EUI) and Ivan Petrella (Collegio Carlo Alberto).\\[.2em]
\textit{Abstract:} US inflation risk is non-symmetric and varies considerably over time. Monetary and fiscal policies along with non-policy factors, such as unit labor costs, long-run interest rates, the unemployment gap, and commodity prices, are key drivers of the inflation risk. Macroeconomic predictors affect the long-run mean of inflation chiefly by influencing the shape and the skewness of the predictive distribution of long-run inflation. Inflation stabilization requires periodic revisions to the monetary and fiscal framework to counterbalance persistent shifts in the inflation risk. Failing to offset the inflation risk led to the large upside inflation risk of the 1960s and the 1970s. Our findings suggest that the Phillips curve is nonlinear and its slope is affected by policy and non-policy factors that have bearings on short-term volatility and risk of inflation.\\[.5em]

\textit{\textbf{Mixed Frequency Functional VARs for Nowcasting and Structural Analysis}},\\ with Gary Koop (Univeristy of Strathclyde), Stuart McIntyre (Univeristy of Strathclyde) and James Mitchell (FRB Cleveand).\\[.2em]
\textit{Abstract:} We propose a functional-Vector Autoregressive model (fVAR) to nowcast the dynamics of the whole income distribution in the United Kingdom. British survey data about household income are published by the Office for National Statistics (ONS) with considerable delay, making them unappealing for policy evaluation. Our approach produces accurate predictions of past, current and future income distributions. We introduce a framework to rank the predictive ability of forecasting models when the target object is a full density, rather than a single realization. Based on this novel loss-function, we establish that out fVAR provides superior forecasting accuracy with respect to competing models. Our model further allows to carry out structural analysis on the income distribution within a traditional VAR setting.\\[.5em]

\textit{\textbf{Real-Time Forecasting with High-Frequency Seasonal Patterns}},\\ with Ana Galv\~ao (Bloomberg LP) and Ivan Petrella (Collegio Carlo Alberto).\\[.2em]
\textit{Abstract:} In this paper, we propose a novel, comprehensive approach to interpolate low-frequency official statistics from high-frequency data. Differently from standard mixed-frequency dynamic factor models, commonly used for nowcasting, we leverage on recent developments in nowcasting modeling to build a methodology that can easily deal with outlier detection, complex calendar patterns and temporal disaggregation. We deploy the new methodology to introduce a new weekly tracker for real activity in the United Kingdom based on the several new, high-frequency data provided by the Office for National Statistics (ONS). Results suggest that these new data sources, when properly managed via our model, introduce significant improvements in the predictive accuracy of traditional nowcasting models, generally based on lower-frequency data, in terms of both point and density forecasts.

\section{Presentations}
\begin{itemize}
    \item[2025:] HM Treasury*, University of Strathclyde, ESCoE Conference on Economic Measurement (King's College London), Bank of Finland, 33rd Annual Symposium of the Society for Nonlinear Dynamics and Economics (University of San Antonio, Texas). 
    \item[2024:] Workshop Empirical Monetary Economics (OFCE, Paris), UNSW-ESCoE Conference on Economic Measurement (University of New South Wales, Sydney), The Frontier of Monitoring and Forecasting Macroeconomic and Financial Risk (SOFiE, National Bank of Belgium), 31st Annual Symposium of the Society for Nonlinear Dynamics and Economics (University of Padova), ESCoE Conference on Economic Measurement Conference (Univeristy of Manchester), University of Verona.
    \item[2023:] Federal Reserve Bank of Chicago, 12th European Central Bank Conference on Forecasting Techniques, Society for Financial Econometrics (Sungkyunkwan University, Seoul), International Association for Applied Econometrics (BI Norwegian Business School), Money, Macro \& Finance Network, 5th International Workshop in Financial Econometrics (Bahia, Brazil), Bank of England, Centre for Macroeconomics (LSE), 17th International Conference on Computational and Financial Econometrics (HTW Berlin). 
    \item[2022:] CEBRA (Pompeu Fabra University), ECB, RCEA Conference on Recent Developments in Economics, Econometrics and Finance (University of Cyprus), Fulcrum Asset Management.
    \item[2021:] Warwick Business School, Economics Statistics Center of Excellence, International Association for Applied Econometrics (Erasmus School of Economics), \nth{7} RCEA Time Series Workshop, International Symposium of Forecasters, \nth{11} RCEA Money, Macro and Finance Conference, NBER-NSF SBIES (University of St. Louis), European Economic Association (University of Copenhagen), \"Orebro University.
    \item[2020:] University of Warwick, 28th Annual Symposium of the Society for Nonlinear Dynamics and Econometrics (Zagreb University), University of Cyprus, Conference on Real-Time Data Analysis, Methods and Applications (FRB Philadelphia), 2nd Vienna Workshop on Economic Forecasting 2020 (IHS), EC\^2 confecerence (CREST \& ESSEC).
    \item[2019:] 13th International Conference on Computational and Financial Econometrics (University of London).
\end{itemize}
*: scheduled.

\section{Referee activity}
American Economic Review: Insights, Journal of Applied Econometrics, Journal of Economic Dynamics and Control, International Journal of Forecasting, The Manchester School, International Review of Financial Analysis
\section{Scholarships and Honors}
\begin{tabular*}{\linewidth}{@{}l@{\extracolsep{\fill}}l}
WBS bursary, Warwick Business School, University of Warwick & 2017 - 2021\\
Award for Outstanding Contribution to Teaching, Warwick Business School & 2020 \& 2021\\
Particularly deserving “Giorgio Mortara” candidate, Bank of Italy & 2017\\
 C.S.R. Pettinari scholarship & 2017\\
\end{tabular*}
\section{Teaching experience}
\begin{tabular*}{\linewidth}{@{}l@{\extracolsep{\fill}}l}
\textbf{Empirical finance} (MSc), Warwick Business School & 2018 - 2021\\[.25em]
\multicolumn{1}{l}{Teaching assistant for Dr. Daniele Bianchi (2018 - 2019)}\\
\multicolumn{1}{l}{Teaching assistant for Dr. Ganesh Viswanath-Natraj (2019 - 2021)}\\
\\[.25em]
\textbf{Research methods} (MSc), Warwick Business School \\[.25em]
\multicolumn{1}{l}{Teaching assistant for Prof. Roman Kozhan (2018 - 2019)}\\
\multicolumn{1}{l}{Teaching assistant for Dr. Gi H. Kim (2010 - 2021)}\\
\\[.25em]
\textbf{Time series} (BSc), Economics dept., University of Warwick & 2018 - 2019 \\[.25em]
\multicolumn{1}{l}{Teaching assistant for Dr. Alexander Karalis Isaac} \\
\\[.25em]
\textbf{Econometrics} (MSc), Warwick Business School & 2019 - 2020\\[.25em]
\multicolumn{1}{l}{Teaching assistant for Prof. Gianna Boero and Dr. Thomas Martin }\\
\\[.25em]
\textbf{Econometrics} (MSc), Economics dept.,  University of Warwick & 2020 - 2021\\[.25em]
\multicolumn{1}{l}{Teaching assistant for Prof. Manuel Bagues}\\
\\[.25em]
\textbf{Applied Multiple Regression Analysis} (PhD), Warwick Business School & 2021\\[.25em]
\multicolumn{1}{l}{Teaching assistant for Prof. Ana Galv\~ao}\\
\end{tabular*}

\section{Advanced training}
\begin{tabular*}{\linewidth}{@{}l@{\hspace{.08\linewidth}}r}
\textbf{Nowcasting \& Models for Mixed Frequency Data} & \sc{Jul} 2021\\
\multicolumn{1}{l}{
International Institute of Forecasters, \nth{4} annual forecasting summer school}\\
\multicolumn{1}{l}{Lecturer: M. Marcellino (Bocconi University)}\\&\\
\textbf{Recent Development in Financial Econometrics} & \sc{Jul} 2018\\
\multicolumn{1}{l}{Italian Econometric Society summer school}\\
\multicolumn{1}{l}{Lecturers: A. Patton (Duke University) and K. Sheppard (Oxford University)}\\&\\
\textbf{Time Series Econometrics} & \sc{May} 2017\\
\multicolumn{1}{l}{Lecturer: A. C. Harvey (Cambridge University)}\\&\\
\textbf{Bayesian Methods for Macroeconomics} & \sc{Apr} 2017\\
\multicolumn{1}{l}{Lecturer: G. Koop (Strathclyde University)}\\&\\
\textbf{Bayesian Econometrics} & \sc{Mar} 2017\\
\multicolumn{1}{l}{Lecturer: M. D. Weeks (Cambridge University)}\\&\\
\textbf{Can you speak Matlab?} & \sc{Mar} 2017\\
\multicolumn{1}{l}{Working with Time and Frequecy in Matlab}\\&\\
\textbf{Can you speak Matlab?} & \sc{Mar} 2016\\
\multicolumn{1}{l}{Solving optimization problems with Matlab}\\&\\
\end{tabular*}

\section{Computer Skills}
\begin{tabular}{ll}
Advanced Knowledge:& \sc{Matlab}, \sc{STATA}, \sc{\LaTeX}, \sc{Beamer}, \sc{Office package}\\
Intermediate Knowledge:& \sc{R}, \sc{Phyton}, \sc{Julia}, \sc{Mathematica}, \sc{Dynare}. \sc{SQL}\\
\end{tabular}

\section{Languages}
\textsc{Italian} (native), \textsc{English} (fluent), \textsc{Spanish} (intermediate), \textsc{French} (basic)
%\section{Interests}
%Travelling, Scuba Diving, Mountaineering, Wildlife, Ecology, Drums
\section{References}
\begin{tabular}{lll}
\begin{minipage}[t]{0.45\textwidth}
Prof.\ \sc{Ivan Petrella}\\
\normalfont Collegio Carlo Alberto\\
\normalfont University of Turin\\
\Letter\ \href{mailto:ivan.petrella@carloalberto.org}{\normalfont ivan.petrella@carloalberto.org}
\end{minipage}
&
\begin{minipage}[t]{0.45\textwidth}
Prof.\ \sc{Leonardo Melosi} \\
\normalfont Department of Economics\\
\normalfont European University Institute \\
\Letter\ \href{mailto:leonardo.melosi@warwick.ac.uk}{\normalfont leonardo.melosi@eui.eu}
% \normalfont Senior Economist and Economic Advisor\\
% \normalfont Federal Reserve Bank of Chicago  \\
% \Letter\ \href{mailto:leonardo.melosi@chi.frb.org}{\normalfont leonardo.melosi@chi.frb.org}
\end{minipage}
\\[5em]
% \begin{minipage}[t]{0.45\textwidth}
% Dr.\ \sc{Daniele Bianchi} \\
% \normalfont School of Economics \& Finance \\
% \normalfont Queen Mary University of London\\
% \Letter\ \href{mailto:d.bianchi@qmul.ac.uk}{\normalfont d.bianchi@qmul.ac.uk}
% \end{minipage}
\begin{minipage}[t]{0.45\textwidth}
Dr.\ \sc{James Mitchell} \\
\normalfont Vice President, Research Department\\
Federal Reserve Bank of Cleveland\\
\Letter\ \href{mailto:james.mitchell@clev.frb.org}{\normalfont james.mitchell@clev.frb.org}
\end{minipage}
% \end{tabular}
% \begin{tabular}{ll}
% \begin{minipage}[t]{0.45\textwidth}
% Prof.\ \sc{Ivan Petrella}\\
% \normalfont Warwick Business School\\
% \normalfont University of Warwick\\
% \Letter\ \href{mailto:ivan.petrella@wbs.ac.uk}{\normalfont ivan.petrella@wbs.ac.uk}
% \end{minipage}
% &
% \begin{minipage}[t]{0.45\textwidth}
% Prof.\ \sc{Leonardo Melosi} \\
% \normalfont Department of Economics\\
% \normalfont University of Warwick \\
% \Letter\ \href{mailto:leonardo.melosi@warwick.ac.uk}{\normalfont leonardo.melosi@warwick.ac.uk}
% % \normalfont Senior Economist and Economic Advisor\\
% % \normalfont Federal Reserve Bank of Chicago  \\
% % \Letter\ \href{mailto:leonardo.melosi@chi.frb.org}{\normalfont leonardo.melosi@chi.frb.org}
% \end{minipage}
% \\[5em]
% % \begin{minipage}[t]{0.45\textwidth}
% % Dr.\ \sc{Daniele Bianchi} \\
% % \normalfont School of Economics \& Finance \\
% % \normalfont Queen Mary University of London\\
% % \Letter\ \href{mailto:d.bianchi@qmul.ac.uk}{\normalfont d.bianchi@qmul.ac.uk}
% % \end{minipage}
% \begin{minipage}[t]{0.45\textwidth}
% Dr.\ \sc{James Mitchell} \\
% \normalfont Vice President, Research Department\\
% Federal Reserve Bank of Cleveland\\
% \Letter\ \href{mailto:james.mitchell@clev.frb.org}{\normalfont james.mitchell@clev.frb.org}
% \end{minipage}
&
\begin{minipage}[t]{0.45\textwidth}
Dr.\ \sc{Ana Beatriz Galv\~{a}o} \\
\normalfont Global Modelling Team\\
\normalfont Bloomberg Economics\\
\Letter\ \href{mailto:ana.b.galvao@pm.me}{\normalfont ana.b.galvao@pm.me}
\end{minipage}
\end{tabular}

\vfill
{\begin{center}\color{black!30} Andrea De Polis, \today\end{center}}

\end{document}
